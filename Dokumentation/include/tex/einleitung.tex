\section{Einleitung}
Für Fördervereine der freiwilligen Feuerwehr ist es eine massive Kostenbelastung, wenn bereits wenige hundert Euro jedes Jahr zu bezahlen sind. Die Träger der Feuerwehren berufen sich bezüglich der Alarmierung per Smartphone darauf, dass bereits durch analoge oder digitale Meldeempfänger ausreichend vorgesorgt ist und die Einsatzkräfte adäquat alarmiert werden.

In der Feuerwehr Kusel wurde beispielsweise eine\\ firEmergency\cite{Alamos:FE2} Installation betrieben, die durch den Förderverein finanziert wurde. Das heißt Spenden, Mitgliederbeiträge und Veranstaltungserlöse werden genutzt, um eine grundlegende Anforderung sicherzustellen. Ein Kostenvergleich zwischen einer aktuellen {firEmergency 2} Installation und der BOSCall Variante folgt in Kapitel~\ref{sec:kostenvergleich}.

In Kapitel~\ref{sec:funktionsweise} wird auf die Funktionsweise der Alarmierung mittels BOSCall eingegangen. Die beiden Alarmetappen, die dabei durchlaufen werden, werden in dem Kapitel näher erläutert. Anschließend dient Kapitel~\ref{sec:problemstellungen} dazu, verschiedene Problemstellungen und deren Lösungen zu erörtern, die bei der Implementierung der App aufgetreten sind.

Kapitel~\ref{sec:installation} beschreibt im Anschluss die Schritte, die erforderlich sind, um die App in Betrieb zu nehmen. Abschließend wird in Kapitel~\ref{sec:fazit} ein Fazit über gezogen. Dabei wird insbesondere auf die erreichten und nicht erreichten Ziele der App eingegangen.